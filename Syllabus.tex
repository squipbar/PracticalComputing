
%Input preamble
%Style
\documentclass[11pt]{article}
\usepackage[top=1in, bottom=1in, left=1in, right=1in]{geometry}
\parindent 22pt

%Packages
\usepackage{amsmath}
\usepackage{amsfonts}
\usepackage{amssymb}
\usepackage{bm}
\usepackage[table]{xcolor}
\usepackage{tabu}
\usepackage{makecell}
\usepackage{longtable}
\usepackage{multirow}
\usepackage[normalem]{ulem}
\usepackage{etoolbox}
\usepackage{graphicx}
\usepackage{tabularx}
\usepackage{ragged2e} 
\usepackage{booktabs}
\usepackage{caption}
\usepackage[none]{hyphenat}
\usepackage{fixltx2e}
\usepackage{threeparttablex}
\usepackage[capposition=top]{floatrow}
\usepackage{subcaption}
\usepackage{pdfpages}
\usepackage{pdflscape}
\usepackage{natbib}
\definecolor{maroon}{HTML}{990012}
\usepackage[colorlinks=true,linkcolor=maroon,citecolor=maroon]{hyperref}
\usepackage{tikz}
\usetikzlibrary{shapes}
\usepackage{setspace}

%Functions
\DeclareMathOperator{\cov}{Cov}
\DeclareMathOperator{\var}{Var}
\DeclareMathOperator{\plim}{plim}

%Math Environments
\newtheorem{theorem}{Theorem}[section]
\newtheorem{claim}[theorem]{Claim}
\newtheorem{assumption}[theorem]{Assumption}
\newtheorem{definition}[theorem]{Definition}
\newtheorem{hypothesis}[theorem]{Hypothesis}
\newtheorem{property}[theorem]{Property}
\newtheorem{example}[theorem]{Example}
\newtheorem{exercise}[theorem]{Exercise}
\newtheorem{condition}[theorem]{Condition}
\newtheorem{remark}[theorem]{Remark}
\newenvironment{proof}{\paragraph{Proof:}}{\hfill$\square$}

%Commands
\newcommand\independent{\protect\mathpalette{\protect\independenT}{\perp}}
\def\independenT#1#2{\mathrel{\rlap{$#1#2$}\mkern2mu{#1#2}}}
\newcommand{\overbar}[1]{\mkern 1.5mu\overline{\mkern-1.5mu#1\mkern-1.5mu}\mkern 1.5mu}
\newcommand{\equald}{\ensuremath{\overset{d}{=}}}

%Logo
%\AddToShipoutPictureBG{%
 % \AtPageUpperLeft{\raisebox{-\height}{\includegraphics[width=1.5cm]{uchicago.png}}}
%}




\begin{document}

\title{\textbf{Practical Computing for Economists}\footnote{This course does not count as a general distribution requirement and does not assign other grade than ``R''. }}
\author{Spring 2015}
\date{First Draft: July 24 , 2014 \\ This Draft: \today}
\maketitle

\begin{abstract}
\noindent The objective of this course is to provide students with the basic knowledge economists need when performing numerical and data analysis. Various programs will be reviewed, particularly focusing on computational advantages and time allocation trade-offs when solving typical problems economists face (e.g. numerical integration, likelihood maximization, bootstrap standard errors). The course is entirely practical and based on knowledge from professors, classes, and autodidact training. It must be noted that this is not a theoretical course on convergence properties or alike nor a class on how computer scientists may code economics problems. Instead, the main goal is to provide economists with the essential computational toolkit to conduct research.
\end{abstract}


\section{JE}

\section{JB}

\section{PB}

\section{BJS}

\section{Structural Econometrics with Implementations in Julia}
\noindent \textbf{Host}: Jorge L. Garc\'{i}a.\\
\noindent \textbf{Time}: two weeks.\\
\noindent \textbf{Description}:  I  briefly describe what are structural models in Economics. I distinguish structural and non-structural estimation as approaches that enable to answer different research questions. Also, I discuss parametric and non-parametric assumptions as auxiliary means that help to recover parameters enabling to answer these research questions. I buid on \citet{keane2011structural} but expand by providing a while range of computational implementations of their baseline model --a classic discrete choice dynamic programming model of women's labor supply. I use Julia in my implementations and provide some Python spinets for the sake of comparison.

\clearpage
\bibliographystyle{chicago}
\bibliography{BibtexFiles/Syllabus}
\clearpage

\end{document}